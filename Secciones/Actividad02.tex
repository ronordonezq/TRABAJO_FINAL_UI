 \section{Marco Teorico} 
\subsection{Modelo de Negocios Canvas}
\begin{enumerate}[1.]
	
	\item Modelo de Negocios Canvas
	\\
	\\
	 Para el desarrollo de nuestro modelo de negocios canvas, algunas personas piensan en tener una buena idea, una idea perfecta. Sin embargo, muchos ni siquiera se han tomado el tiempo de validarla con un grupo de personas con necesidades y gustos distintos a los propios. En 2010 Alex Osterwalder diseño el Business Model Canvas: un formato que visualiza el modelo de negocio en una sola hoja con 9 divisiones. Este método resulta en un documento que ofrece una visión global detallada de la idea de negocio, mostrando claramente las interconexiones entre los diferentes elementos.\\
Estas 9 divisiones ayudan a separar todo el flujo de trabajo de la idea que tenemos en mente. Poco a poco vamos a ir estructurándola y dándole sentido. Para tener éxito al desarrollar este modelo para una idea propia, lo primero es conocer y tener un concepto detallado de lo que significa cada sección:\\
		\begin{enumerate}[a)]
			\item Segmento de Cliente \\
			\\
			Se refiere a los grupos de personas a los que se quiere ofrecer el producto o servicio. Son la base del negocio, así que es ideal conocerlos muy bien. Para entenderlos es importante recordar que, en algunos casos, un cliente es quien paga y un usuario es aquel que sólo consume el producto o utiliza el servicio. Un ejemplo práctico sería el de un niño al que su padre le regala un videojuego: el papá es el cliente y el niño es el usuario final.\\

			\item Propuesta de Valor \\
			\\
			Se trata del pain statement que solucionamos para el cliente y cómo le damos respuesta con el producto y/o servicio. La iniciativa debe sobresalir en el mercado, nos debe hacer notar. Este es el valor agregado que nos llevaría al éxito.\\
			\\
			\item Canales de Distribucion \\
			\\
			Se centra en determinar cómo comunicar, alcanzar y entregar la propuesta de valor a los clientes. Estamos en siglo XXI y podemos utilizar el internet para este propósito a través de un sitio web, redes sociales, mailing, centro comercial, etc.\\
			\\
			\item Relaciones con el Cliente \\
			\\
			Es uno de los aspectos más críticos en el éxito del modelo de negocio y uno de los más complejos de hacer tangible. Se pueden establecer diferentes tipos de relaciones, dependiendo del segmento o tipo de cliente. Las relaciones humanas son sumamente importantes. Un ejemplo práctico: redes sociales.\\
			\\
			\item Flujo de Ingreso \\
			\\
			Representa la forma en la que se generan los ingresos por cada cliente. La obtención de ingresos puede ser directa o indirecta, en un sólo pago o recurrente. Debemos analizar muy bien y preguntarnos por qué pagaría el cliente por este producto o servicio y cuál sería su método de pago. En algunos casos se puede utilizar el modelo de pago fremium, que consiste en ofrecer el servicio de manera gratuita (o con uso limitado) por un tiempo. Para, después, ofrecer una mejor experiencia con la condición de pago.\\
			\\
			\item Recursos Clave \\
			\\
			Los recursos más importantes, necesarios para el funcionamiento del negocio. Se clasifican por tipo, cantidad e intensidad. Debemos saber qué vamos a necesitar para llevar a cabo el proyecto. Esto incluye la mano de obra: es importante buscar a las personas adecuadas con el perfil indicado.\\
			\\
			\item Actividades Clave \\
			\\
			Para entregar la propuesta de valor se deben desarrollar una serie de actividades clave internas como:\\
			\\
			- Definir procesos de producción\\
			- Marketing\\
			- Tiempos de desarrollo\\
			- Cronograma de actividades detallado con tiempos y fechas\\
			\\
			\item Red de Asociados\\
			\\
			Se definen las alianzas necesarias para ejecutar el modelo de negocio con garantías. Estas deben complementar las capacidades y optimizar la propuesta de valor: la co-creación es imprescindible hoy en día en los negocios. Una buena idea requiere de una inversión y para ello podemos realizar alianzas con inversionistas o empresas capaces de aportar. En ocasiones, las alianzas no se hacen sólo por obtener un fin económico; sino que el proyecto puede generar beneficios para ambas partes.\\
			\\			
			\item Estructura de Costos\\
			\\
			Describe todos los costos necesarios al operar el modelo de negocio. Se trata de conocer y optimizar el capital para intentar diseñar un modelo de negocio sostenible, eficiente y escalable. Después de seguir este modelo, tendremos una idea de negocio bastante compacta que puede seguir desarrollándose todos los días. Una recomendación es imprimirlo y pegarlo en un lugar visible para que cada que surja una nueva idea, se pueda agregar en la sección que corresponda. Estas ideas pueden validarse en el mercado y desecharse o complementarse. Con el paso del tiempo, tendremos un proyecto completo que puede pasar a la etapa de elaboración o ejecución con plena garantía de que si trabajamos con dedicación, constancia y sacrificio se convertirá en un gran negocio, producto o servicio, innovador y sostenible.\\

		\begin{center}
                    \includegraphics[scale=0.60]{./Imagenes/ang_1}
                    \end{center}


\subsection{Desarrollo del Modelo Canvas con RENIEC}		
		\end{enumerate}
\begin{enumerate}[a)]
			\item Segmento de Cliente \\
Genera documentos en formato excel para visualizar la informacion que desea.
	        \begin{center}
                    \includegraphics[scale=0.30]{./Imagenes/a.png}
                    \end{center}
\item Propuesta de Valor \\
		 \begin{center}
	En los documentos excel se visualisan la informacion que desea ver. como en este ejemplo que tenemos la informacion de los registros civiles del 2016.
                    \includegraphics[scale=0.30]{./Imagenes/b.png}
                    \end{center}

\item Canales de Distribucion \\
La RENIEC cuenta con un sitio web donde muestra toda su informacion
 \begin{center}
                    \includegraphics[scale=0.30]{./Imagenes/c.png}
                    \end{center}
\item Relaciones con el Cliente \\
Cuentan con diferentes redes sociales para que el cliente se informe
 \begin{center}
                    \includegraphics[scale=0.30]{./Imagenes/d.png}
\includegraphics[scale=0.30]{./Imagenes/e3.png}
                    \end{center}
\item Flujo de Ingreso \\
 \begin{center}
Cuentan con varios servicios para hacer consultas gratis
                    \includegraphics[scale=0.30]{./Imagenes/e.png}
\includegraphics[scale=0.30]{./Imagenes/e1.png}
\includegraphics[scale=0.30]{./Imagenes/e2.png}
                    \end{center}
\item Recursos Clave \\
 \begin{center}
Ofrecen diferentes empleos para que que las personas puedan postular
                    \includegraphics[scale=0.30]{./Imagenes/f.png}
                    \end{center}
\item Actividades Clave \\
 \begin{center}

                    \includegraphics[scale=0.30]{./Imagenes/g.png}
                    \end{center}
\end{enumerate}	
		
\subsection{Definicion Qlik Sense} 

              \item ¿Qu\'e es Qlink Sense?. \\
               
	   

                     \begin{center}
                    \includegraphics[scale=0.60]{./Imagenes/qlik_sense.png}
                    \end{center}
                
                   Qlik Sense es una aplicaci\'on de visualizaci\'on y descubrimiento de datos gobernada, basada en servidor, ideal para las necesidades anal\'iticas de grupos, departmentos o toda una organizaci\'on. Los usuarios de negocio obtienen un an\'alisis de datos potente, flexible y personalizado y colaboraci\'on en cualquier dispositivo, a la vez que se adhieren a unas pol\'iticas de gobierno y seguridad centralizada de datos.\\
                  \\ 
              \item ¿Como podemos hacer?\\
               \\ 
               La mayor\'ia de productos de Business Intelligence (BI) ayudan a las personas a responder preguntas que ya se comprenden de antemano. Pero ¿qu\'e ocurre con las preguntas que se nos van ocurriendo despu\'es o sobre la marcha? ¿Ese tipo de preguntas que surgen tras leer un informe o visualizar un grafico? Con la experiencia asociativa de Qlik Sense, podemos hacer todas las preguntas que se nos ocurran y responderlas una tras otra, avanzando por nuestra propia ruta hacia el conocimiento. Con Qlik Sense podemos explorar los datos libremente, mediante simples clics de rat\'on, aprendiendo y profundizando en cada etapa del camino y descubriendo nuevas rutas de exploraci\'on basadas en nuestros propios descubrimientos.
              \\
              \\
               \\
               \\
              \item Utilidades \\
            \\
A trav\'es de visualizaciones inteligentes puedes: \\
             
                \begin{center}
                    \includegraphics[scale=0.60]{./Imagenes/caracteristicas.png}
                 \end{center}

        \begin{itemize}
         \item Transmitir el significado de los datos con visualizaciones inteligentes, innovadoras, completamente interactivas y con capacidad de reacci\'on.\\
         \item Explorar en cualqxuier direcci\'on: encuentrar datos e informaci\'on valiosa que las herramientas jerarquicas y basadas en consultas no detectan.\\
         \item Lograr una flexibilidad absoluta: solo tienes que escribir lo que necesites para encontrar información relacionada y ver datos relacionados en todo el conjunto de datos.\\
          \item Explorar m\'ultiples fuentes de datos: conectar y visualizar datos de varias fuentes para una vista m\'as exhaustiva.\\
          \item Narraci\'on de datos detallada: colaborar y compartir la informaci\'on extraida del an\'alisis visual. Comunicar mejor los hallazgos a su equipo. Moverte directamente entre historias y an\'alisis en directo para responder a preguntas y acelerar la toma de decisiones.\\
        \end{itemize}

         \item  Caracteristicas 
    
       \begin{description}
            \item[Multifuente:] Se conecta con m\'ultiples fuentes de datos, incluyendo entradas de datos en tiempo real, a fin de proporcionar unas vistas aun m\'as exhaustivas, sin comprometer el rendimiento de las aplicaciones.\\
            \item[Colaborativo:] La funcionalidad de Qlik Sense le permitir\'a una narraci\'on de datos facil con la que podra compartir el análisis de una forma visual, comunicar sus hallazgos a los equipos y colaborar con mayor eficacia.\\
            \item[AutoServicio:] Cualquier usuario puede crear sus propias visualizaciones de datos, sus cuadros de mando, al tiempo que ofrece a TI la confianza de estar diseñando unas librerías seguras y consistentes y unos datos bien gobernados.\\
            \item[DragandDrop:] Las visualizaciones inteligentes, en combinaci\'on con los datos Qlik patentados de su motor de indexaci\'on, descubren todas las relaciones entre las dimensiones de datos, revelando conocimientos que habr\'ian permanecido ocultos en los modelos tradicionales de datos basados en consultas y jerarqu\'ias. Datos, informacion y conocimiento.\\

        \end{description}
               


    
\end{enumerate}
