 \section{Resumen} 
El Modelo Canvas es una herramienta estratégica destinada —no exclusivamente— a emprendedores es desarrollada por Alexander Osterwalder (teórico austriaco, nacido en 1974) en su libro coescrito con Yves Pigneur (informático belga, profesor en el HEC en Lausana, nacido en 1954) Business Model Generation (2010), convertido en un auténtico superventas.

El objetivo del modelo Canvas es transformar las ideas en proyectos innovadores y competitivos en el mercado. Para ello, los autores invitan a cada empresa que la utiliza a reflexionar acerca del valor que crea, tanto para los clientes como para las propias empresas.

Este modelo está particularmente adaptado a las personas que evolucionan en pequeñas empresas o en empresas emergentes cuya estructura no tiene una fuerte jerarquía: en realidad, este lienzo propone un enfoque más sistémico que la mayor parte de los modelos tradicionales, articulando los diferentes elementos constituyentes de la empresa.


El BMC se inscribe en la tendencia visual  design thinking, es decir, permite crear un sistema visual accesible, legible y comprensible para todos gracias a su procedimiento no lineal. Este lienzo en un soporte con el que los emprendedores piensan y construyen su modelo económico en una única página: organizan fácilmente sus ideas en la «plantilla» con casillas para pasar más rápida y eficazmente a la acción. El hecho de ofrecer una visión de conjunto a los modelos en construcción favorece la definición clara de las prioridades y de los planes de acción concretos que se deben llevar a cabo, así como un enfoque creativo y adaptable, lo que simplifica enormemente la elaboración futura de un plan de negocios. Esta herramienta también mejora los intercambios con los clientes y da un verdadero impulso a la comunicación entre trabajadores.